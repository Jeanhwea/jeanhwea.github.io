% Created 2023-02-02 Thu 11:56
% Intended LaTeX compiler: xelatex
\documentclass[presentation]{beamer}
\usepackage{graphicx}
\usepackage{longtable}
\usepackage{wrapfig}
\usepackage{rotating}
\usepackage[normalem]{ulem}
\usepackage{amsmath}
\usepackage{amssymb}
\usepackage{capt-of}
\usepackage{hyperref}
\usepackage[scheme=plain]{ctex}
\usetheme{Madrid}
\author{Jinghui Hu}
\date{\textit{<2023-02-02 Thu 11:21>}}
\title{GroupBy语句实现原理探索}
\hypersetup{
 pdfauthor={Jinghui Hu},
 pdftitle={GroupBy语句实现原理探索},
 pdfkeywords={},
 pdfsubject={},
 pdfcreator={Emacs 28.2 (Org mode 9.5.5)},
 pdflang={English}}
\begin{document}

\maketitle
\begin{frame}{Outline}
\tableofcontents
\end{frame}



\section{数据准备}
\label{sec:org827e8a0}
\begin{frame}[label={sec:org039b5e1},fragile]{建表}
 \begin{verbatim}
create table tb_data_01 (
  id int primary key,
  a int,
  b int,
  index (a)
) engine = innodb;
\end{verbatim}

\begin{verbatim}
select * from tb_data_01 limit 10;
\end{verbatim}

\begin{center}
\begin{tabular}{rrr}
id & a & b\\
\hline
1 & 1 & 1\\
2 & 2 & 2\\
3 & 3 & 3\\
4 & 4 & 4\\
5 & 5 & 5\\
6 & 6 & 6\\
7 & 7 & 7\\
8 & 8 & 8\\
9 & 9 & 9\\
10 & 10 & 10\\
\end{tabular}
\end{center}
\end{frame}
\begin{frame}[label={sec:org4becff4},fragile]{插入数据}
 \begin{verbatim}
delimiter ;;
  create procedure idata()
  begin
    declare i int;
    set i=1;
    while(i<=1000)do
      insert into tb_data_01 values(i,i,i);
      set i=i+1;
    end while;
  end;;
delimiter ;

call idata();
\end{verbatim}
\end{frame}
\end{document}